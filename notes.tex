\documentclass[12pt]{article}
\usepackage[utf8]{inputenc}
\usepackage[margin=1in, top=1.5in]{geometry}
\usepackage{fancyhdr}
\usepackage{amsthm}
\usepackage{amssymb}

\pagestyle{fancy}
\fancyhf{}

\lhead{Algebra}
\rhead{Notes}

\begin{document}

\section*{Chapter 11: Rings}

\begin{itemize}
  \item \textbf{Ring}: Algebraic structure closed under addition, subtraction, and multiplication (not division).
  \item \textbf{Subring}: Subset which is closed under addition, subtraction, multiplication and which contains 1.
  \item \textbf{Gauss integers}: The complex numbers of the form $\textit{a} + \textit{bi}$ where \textit{a} and \textit{b} are integers form a subring of $\mathbb{C}$ that we denote by $\mathbb{Z}$[i] = \{$\textit{a} + \textit{bi} \mid \textit{a}, \textit{b} \in \mathbb{Z}$\}.
\end{itemize}

\subsection*{Problem 2}
\textbf{Show that if M is the closed interval [a,b] and p is not in M, then p is not a limit point of M.}
\begin{proof}
Since p is not in M, it is not between or equal to a and b, and therefore any interval that contains p is not within M.
\end{proof}

\end{document}
