\documentclass[12pt]{article}
\usepackage[utf8]{inputenc}
\usepackage[margin=1in, top=1.5in]{geometry}
\usepackage{fancyhdr, amsthm, amssymb, amsmath}
\usepackage{tikz,mathtools}

\setlength\parindent{0pt}
\pagestyle{fancy}
\fancyhf{}
\lhead{Algebra}
\rhead{Notes}
\rfoot{\thepage}

\begin{document}

\section*{Chapter 11: Rings}

\subsection*{11.1: Definition of a Ring}
\begin{itemize}
  \item \textbf{Ring}: A \textit{ring} $R$ is a set with two laws of composition $+$ and $\times$, called addition and multiplication, that satisfy these axioms:
  \begin{enumerate}
    \item[(a)] With the law of composition $+$, $R$ is an abelian group that we denote by $R^+$; its identity is denoted by 0.
    \item[(b)] Multiplication is commutative and associative, and has an identity denoted by 1.
    \item[(c)] \textit{Distributive law}: For all $a, b$ and $c$ in $R$, $(a + b)c = ac + bc$.
  \end{enumerate}
  \begin{itemize}
    \item \textbf{Subring}: Subset which is closed under addition, subtraction, multiplication and which contains 1.
    \item \textbf{Non-commutative Ring}: Satisfies all of the above axioms, except for the commutative law for multiplication.
  \end{itemize}
  \item \textbf{Gauss integers}: The complex numbers of the form $a + bi$ where $a$ and $b$ are integers form a subring of $\mathbb{C}$ that we denote by $\mathbb{Z}$[i] = \{$a + bi \mid b, b \in \mathbb{Z}$\}. Its elements are points of a square lattice in the complex plane.
  \begin{itemize}
    \item \textbf{$\mathbb{Z}$[$\alpha$] subring}: Contains every complex number $\beta$ = $a_n\alpha^n + \cdots + a_1\alpha + a_0$ where $a_i$ are in $\mathbb{Z}$ and $\alpha$ is a complex number.
    \begin{itemize}
      \item Analogous to the ring of Gauss integers.
      \item Subring generated by $\alpha$
      \item Usually not represented as a lattice in the complex plane
    \end{itemize}
  \end{itemize}
  \item A complex number $\alpha$ is \textbf{algebraic} if it is a root of a (nonzero) polynomial with integer coefficients (i.e. if some expression of the form $a_n\alpha^n + \cdots + a_1\alpha + a_0$ evaluates to 0)
  \begin{itemize}
    \item When $\alpha$ is algebraic there will be many polynomial expressions that represent the same complex number.
  \end{itemize}
  \item If there is no polynomial with integer coefficients having $\alpha$ as a root, $\alpha$ is \textbf{transcendental}
  \begin{itemize}
    \item When $\alpha$ is transcendental, two distinct polynomial expressions represent distinct complex numbers, and the elements of the ring $\mathbb{Z}$[$\alpha$] correspond bijectively to polynomials $p(x)$ with integer coefficients.
  \end{itemize}
  \item A polynomial in $x$ with coefficients in a ring $R$ is an expression of the form
  \begin{center}
    $a_nx^n + \cdots + a_1x + a_0$
  \end{center}
  with $a_i$ in $R$.
  \item \textbf{Zero Ring}: A ring containing only  the element 0.
  \begin{itemize}
    \item A ring $R$ in which the elements 1 and 0 are equal is the zero ring.
  \end{itemize}
  \item \textbf{Unit}: A \textit{unit} of a ring is an element that has a multiplicative inverse (if it exists, it is unique)
  \begin{itemize}
    \item Units in the ring of integers are 1 and -1
    \item Units in the ring of Gauss integers are $\pm$1 and $\pm$i
    \item Units in the ring $\mathbb{R}$[$x$] of real polynomials are the nonzero constant polynomials
    \item The identity element 1 of a ring is always a unit
  \end{itemize}
\end{itemize}

\subsection*{11.2: Polynomial Rings}
\begin{itemize}
  \item \textbf{Formal Polynomial}: A polynomial with coefficients in a ring $R$ is a (finite) linear combination of powers of the variable: $f(x) = a_nx^n + a_{n-1}x^{n-1} + \cdots + a_1x + a_0$ where the coefficients $a_i$ are elements of $R$.
  \begin{itemize}
    \item The set of polynomials with coefficients in a ring $R$ will be denoted $R$[$x$]
    \item Thus $\mathbb{Z}$[$x$] is the set of \textit{integer polynomials}
  \end{itemize}
  \item The \textit{monomials} $x^i$ are considered independent, so if $\exists$ another polynomial with coefficients in $R$, then $f(x) = g(x)$ only if $a_i = b_i$ for all $i = 0, 1, 2, ...$
  \item \textbf{Degree}: The \textit{degree} of a nonzero polynomial (denoted deg $f$) is the largest integer $n$ such that the coefficient $a_n$ of $x_n$ is not zero
  \begin{itemize}
    \item A polynomial of degree zero is called a \textit{constant} polynomial
    \item The zero polynomial is also a constant polynomial, but its degree will not be defined
  \end{itemize}
  \item \textbf{Leading Coefficient}: The nonzero coefficient of highest degree of a polynomial
  \begin{itemize}
    \item \textbf{Monic Polynomial}: Polynomial with a leading coefficient of 1
  \end{itemize}
  \item A polynomial is determined by its vector of coefficients $a_i$: $a = (a_0, a_1, ...)$ where $a_i$ are elements of $R$, all but a finite number zero.
  \item When $R$ is a field, these infinite vectors form the vector space $Z$ with the infinite basis $e_i$. The vector $e_i$ corresponds to the monomial $x_i$, and the monomials form a basis of the space of all polynomials.
  \item \textbf{Addition of polynomials}: $f(x) + g(x) = (a_0 + b_0) + (a_1 + b_1)x + ...$ where $(a_i + b_i)$ is addition in $R$
  \item \textbf{Multiplication of polynomials}: $f(x)g(x) = (a_0 + a_1x + ...)(b_0 + b_1x + ...)$ where $a_ib_j$ are to be evaluated in the ring $R$.
  \item There is a unique commutative ring structure on the set of polynomials $R[x]$ having these properties:
  \begin{itemize}
    \item Additions of polynomials as defined above
    \item Multiplication of polynomials as defined above
    \item The ring $R$ becomes a subring of $R[x]$ when the elements of $R$ are identifies with the constant polynomials
  \end{itemize}
  \item \textbf{Division with Remainder}: Let $R$ be a ring, $f$ is a monic polynomial, and $g$ is any polynomial, both with coefficients in $R$. There are uniquely determined polynomials $q$ and $r$ in $R[x]$ s.t. $g(x) = f(x)q(x) + r(x)$ where $r$ has degree $\geqslant 0$ and $\leqslant f$
  \begin{itemize}
    \item Division with remainder can be done whenever the leading coefficient of $f$ is a unit
    \item If $g(x)$ is a polynomial in $R[x]$ and $\alpha$ is an element of $R$, the remainder of division of $g(x)$ by $x - \alpha$ is $g(\alpha)$. Thus $x - \alpha$ divides $g$ in $R[x]$ iff $g(\alpha) = 0$
  \end{itemize}
  \item \textbf{Monomial}: a formal product of some variables $x_1, ..., x_n$ of the form
  \begin{center}
    $x_1^{i_1}x_2^{i_2}...x_n^{i_n}$
  \end{center}
  where $i_v$ are non-negative integers.
  \begin{itemize}
    \item \textbf{Degree}: the sum $i_1 + \cdots + i_n$, sometimes called \textit{total degree}
    \item \textbf{Multi-index}: an $n$-tuple that can be represented with vector notation e.g.  $i = (i_1, \cdots i_n)$.
    \item A monomial can be written as $x^i$ ($= x_1^{i_1}x_2^{i_2} \cdots x_n^{i_n}$) using multi-index form
    \item The monomial $x^0$ is denoted by 1
  \end{itemize}
  \item With multi-index notation, a polynomial $f(x) = f(x_1, ..., x_n)$ can be written in exactly one way in the form
  \begin{center}
    $f(x) = \sum\limits_ia_ix^i$
  \end{center}
  where $i$ runs through all multi-indices $(i_1, ..., i_n)$, the coefficients $a_i$ are in $R$ and only finitely many of these coefficients are not 0.
  \item \textbf{Homogeneous Polynomial}: A polynomial in which all monomials with nonzero coefficients have degree $d$
\end{itemize}

\subsection*{11.3: Homomorphisms and Ideals}
\begin{itemize}
  \item \textbf{Ring Homomorphism}: A \textit{ring homomorphism} $\phi: R \to R'$ is a map from one ring to another which is compatible with the laws of composition and which carries the unit element 1 of $R$ to the unit element 1 of $R'$ - a map such that for all $a$ and $b$ in $R$,
  \begin{center}
    $\phi(a + b) = \phi(a) + \phi(b), \quad \phi(ab) = \phi(a)\phi(b), \quad and \quad \phi(1) = 1$
  \end{center}
  \begin{itemize}
    \item The map $\phi: \mathbb{Z} \to \mathbb{F}_p$ that send an integer to its congruence class modulo $p$ is a ring homomorphism.
  \end{itemize}
  \item \textbf{Isomorphism}: An \textit{isomorphism} of rings is a bijective homomorphism, denoted $R \approx R'$
  \item Evaluation of real polynomials at a real number $a$ defines a homomorphism
  \begin{center}
    $\mathbb{R}[x] \to \mathbb{R}$, \quad that sends \quad $p(x) \leadsto p(a)$
  \end{center}
  \item \textbf{Substitution Principle}: Let $\phi: R \to R'$ be a ring homomorphism, and let $R[x]$ be the ring of polynomials with coefficients in $R$.
  \begin{enumerate}
    \item[(a)] Let $\alpha$ be an element of $R'$. There is a unique homomorphism $\Phi: R[x] \to R'$ that agrees with the map $\phi$ on constant polynomials, and that send $x \leadsto a$
    \item[(b)] Given elements $\alpha_1, ..., \alpha_n$ of $R'$, there is a unique homomorphism $\Phi: R[x_1, ..., x_n] \to R'$, from the polynomial ring in $n$ variables to $R'$, that agrees with $\phi$ on constant polynomials and that send $x_v \leadsto \alpha_v$, for $v = 1, ..., n$.
  \end{enumerate}
  \item Let $R$ be any ring, and let $P$ be the polynomial ring $R[x]$. One can use the substitution principle to construct an isomorphism
  \begin{center}
    $R[x,y] \to P[y] = (R[x])[y]$
  \end{center}
  This statement is a formalization of the procedure of collecting terms of like degree in $y$ in a polynomial $f(x,y)$. For example:
  \begin{center}
    $x^4y + x^3 - 3x^2y + y^2 + 2 = y^2 + (x^4 - 3x^2)y + (x^3 + 2)$
  \end{center}
  \item Let $x = (x_1, ..., x_m)$ and $y = (y_1, ..., y_n)$ denote sets of variables. There is a unique isomorphism $R[x,y] \to R[x][y]$, which is the identity on $R$ and sends the variables to themselves.
  \item Let $f(x,y)$ and $g(x,y)$ be polynomials in two variables, elements of $R[x,y]$. Suppose that $f$ is a monic polynomial of degree $m$ (grouped by $y$). There are uniquely determined polynomials $q(x,y)$ and $r(x,y)$ such that $g = fq + r$ and $0 \leqslant r(x,y) < m$
  \item There is exactly one homomorphism $\phi: \mathbb{Z} \to R$, defined for $n \geqslant 0$ where $\phi(n) = 1 + \cdots + 1$ (for $n$ terms) and $\phi(-n) = -\phi(n)$
  \item \textbf{Kernel}: The \textit{kernel} of $\phi$ is the set of elements $R$ that map to zero:
  \begin{center}
    ker$\phi = \{s \in R \mid \phi(s) = 0\}$
  \end{center}
  \begin{itemize}
    \item If $s$ is in $ker\phi$, then for every element $r$ of $R$, $rs$ is in $ker\phi$
  \end{itemize}
  \item \textbf{Ideal}: An \textit{ideal} $I$ of a ring $R$ is a nonempty subset of $R$ with these properties:
  \begin{enumerate}
    \item[(a)] $I$ is closed under addition, and
    \item[(b)] If $s$ is in $I$ and $r$ is in $R$, then $rs$ is in $I$
  \end{enumerate}
  \begin{itemize}
    \item \textbf{Principal Ideal}: The ideal formed by multiples of a particular element $a$, also defined as:
    \begin{center}
      $(a) - aR = Ra = \{ra \mid r \in R\}$
    \end{center}
    \item \textbf{Unit Ideal}: The ring $R$ is the principal ideal (1), and is  called the \textit{unit ideal}
    \item \textbf{Zero Ideal}: The principal ideal (0)
    \item \textbf{Proper Ideal}: An ideal that is neither the unit or zero ideal
  \end{itemize}
  \item The kernel of a ring homomorphism is an ideal
  \item An ideal is not a subring unless the ideal $I$ is equal to the whole ring $R$
  \item The ideal \textit{generated by a set of elements} $\{a_1, ..., a_n\}$ of a ring $R$ is the smallest ideal that contains those elements. This ideal is often denoted as $(a_1, ..., a_n)$:
  \begin{center}
    $(a_1, ..., a_n) = \{r_1a_1 + \cdots + r_na_n \mid r_i \in R\}$
  \end{center}
  \item The only ideals of a field are the zero ideal and the unit ideal
  \item A ring that has exactly two ideals is a field
  \item Every homomorphism $\phi: F \to R$ from a field $F$ to a nonzero ring $R$ is injective
  \item The ideals in the ring of integers are the subgroups of $\mathbb{Z}^+$, and they are principal ideals
  \item Every ideal in the ring $F[x]$ of polynomials in one variable $x$ over a field $F$ is a principal ideal. A nonzero ideal $I$ in $F[x]$ is generated by the unique monic polynomial of lower degree that it contains.
  \item Let $f$ be a monic integer polynomial, and let $g$ be another integer polynomial. If $f\mid g$ in $\mathbb{Q}[x]$, $f\mid g$ in $\mathbb{Z}[x]$
  \item \textbf{Greatest Common Divisor}: Let $R$ denote the polynomial ring $F[x]$ in one variable over a field $F$, and let $f$ and $g$ be elements of $R$, not both zero. Their \textit{greatest common divisor} $d(x)$ is the unique monic polynomials that generates the ideal $(f,g)$. It has these properties:
  \begin{enumerate}
    \item[(a)] $Rd = Rf + Rg$
    \item[(b)] $d$ divides $f$ and $g$
    \item[(c)] If a polynomial $e = e(x)$ divides both $f$ and $g$, it also divides $d$
    \item[(d)] There are polynomials $p$ and $q$ such that $d = pf + qg$
  \end{enumerate}
  \item \textbf{Characteristic}: The non-negative integer $n$ that generates the kernel of the homomorphism $\phi: \mathbb{Z} \to R$
  \begin{enumerate}
    \item If $n = 0$, this means that no positive multiple of 1 in $R$ is equal to zero. Otherwise $n$ is the smallest positive integer s.t. "$n$ times 1" is zero in R
  \end{enumerate}
\end{itemize}

\subsection*{11.4: Quotient Rings}
\begin{itemize}
  \item Let $I$ be an ideal of a ring $R$. There is a unique ring structure on the set $\bar{R}$ ($R/I$) of additive cosets of $I$ such that the map $\pi: R \to \bar{R}$ that send $a \leadsto \bar{a} = [a + I]$ (the coset generated with the subgroup $I$) is a ring homomorphism. The kernel of $\pi$ is $I$.
  \begin{itemize}
    \item \textbf{Canonical Map}: $\pi$
    \item \textbf{Quotient Ring}: $\bar{R}$
    \item \textbf{Residue}: The image $\bar{a}$ of $a$
  \end{itemize}
  \item \textbf{Mapping Property of Quotient Rings}: Let $f: R \to R'$ be a ring homomorphism with kernel $K$ and let $I$ be another ideal. Let $\pi: R \to \bar{R}$ be the canonical map from $R$ to $\bar{R} = R/I$
  \begin{enumerate}
    \item[(a)] If $I \subset K$, there is a unique homomorphism $\bar{f}: \bar{R} \to R'$ such that $\bar{f}\pi = f: R \to R/I \to R'$.
    \item[(b)] \textbf{First Isomorphism Theorem}: If $f$ is onto and $I = K$, \bar{f} is an isomorphism.
  \end{enumerate}
  \item \textbf{Correspondence Theorem}: Let $\phi: R \to \mathcal{R}$ be an onto ring homomorphism with kernel $K$. There is a bijective correspondence between the set of all ideals of $\mathcal{R}$ and the set of ideals of $R$ that contain $K$. It is defined as follows:
  \begin{itemize}
    \item If $I$ is an ideal of $R$ and of $K \subset I$, the corresponding ideal of $\mathcal{R}$ is $\phi(I)$
    \item If $\mathcal{I}$ is am ideal of $\mathcal{R}$, the corresponding ideal of $R$ is $\phi^{-1}(\mathcal{I})$
  \end{itemize}
  \item Of the ideal $I$ of $R$ corresponds to the ideal $\mathcal{I}$ of $\mathcal{R}$, the quotient rings $R/I$ and $\mathcal{R/I}$ are naturally isomorphic.
  \item The image of a subgroup is a subgroup
  \item We reinterpret the quotient ring construction when the ideal is principal ($I = (a)$). In this situation, $\bar{R} = R/I$ as the "killing" of $a$ by imposing the relation $a = 0$ on $R$
  \begin{itemize}
    \item Imposing the relation $a = 0$ on $R$ forces us to set $b = b + ra$ for all $b, r$ in $R$
    \item Two elements $b$ and $b'$ of $R$ have the same image in $\bar{R}$ iff $b'$ had the form $b + r_1a_1 + \cdots + r_na_n$ for some $r_i \in R$
  \end{itemize}
\end{itemize}

\subsection*{11.5: Adjoining Elements}
\begin{itemize}
  \item \textbf{Ring Extension}: A ring that contains another ring as a subring
  \item \textbf{Adjoining an Element to a Ring}: We want to adjoin an element $\alpha$ to a ring $R$ and we want $\alpha$ to satisfy the polynomial relation $f(x) = 0$, where
  \begin{center}
    $f(x) = a_nx^n + a_{n-1}x^{n-1} + \cdots + a_ix + a_0$ \quad with $a_i \in R$
  \end{center}
  The solution is $R' = R[x]/(f)$ where $(f)$ is the principal ideal of $R[x]$ generated by $f$.
  \item We let $\alpha$ denote the residue $\bar{x}$ of $x$ in $R'$. Then because the map $\pi: R[x] \to R[x]/(f)$ is a homomorphism,
  \begin{center}
    $\pi(f(x)) = \overline{f(x)} = \bar{a}_n\alpha^n + \cdots + \bar{a}_0 = 0$
  \end{center}
  Where $\bar{a}_i$ is the image in $R'$ of the constant polynomial $a_i$. So $\alpha$ satisfies the relation $f(\alpha) = 0$
  \item Let $R$ be a ring, and let $f(x)$ be a monic polynomial of positive degree $n$ with coefficients in $R$. Let $R[\alpha]$ denote the ring $R[x]/(f)$ obtained by adjoining an element satisfying the relation $f(\alpha) = 0$:
  \begin{enumerate}
    \item[(a)] The set $(1, \alpha, ..., \alpha^{n-1})$ is a \textit{basis} of $R[\alpha]$ over $R$: every element of $R[\alpha]$ can be written uniquely as a linear combination of this basis, with coefficients in   $R$.
    \item[(b)] Addition of two linear combination is vector addition
    \item[(c)] Multiplication of linear combinations is as follows: Let $\beta_1$ and $\beta_2$ be elements of $R[\alpha]$, and let $g_1(x)$ and $g_2(x)$ be polynomials s.t. $\beta_1 = g_1(\alpha)$ and $\beta_2 = g_2(\alpha)$. One divides the product polynomial $g_1g_2$ by $f$, say $g_1g_2 = fq + r$, where the remainder $0 \leqslant r(x) < n$. Then $\beta_1\beta_2 = r(\alpha)$.
  \end{enumerate}
  \item Let $f$ be a \textit{monic} polynomial of degree $n$ in a polynomial ring $R[x]$. Every nonzero element of $(f)$ has degree of at least $n$.
  \item The kernel of $\psi$ (homomorphism that takes $R \to R'$ by restricting the canonical map $\pi$ to just the constant polynomials in $R[x]$) is the set of constant polynomials in the ideal:
  \begin{center}
    ker $\psi = R \cap (f)$
  \end{center}
  ker $\psi$ will likely be zero because $f$ will have positive degree, and we would need to make a polynomial multiple of $f$ have degree zero.
\end{itemize}

\subsection*{11.6: Product Rings}
\begin{itemize}
  \item \textbf{Product Ring}: Let $R$ and $R'$ be rings.
  \begin{enumerate}
    \item[(a)] The product set $R \times R'$ is a ring called the \textit{product ring}, with component-wise addition and multiplication.
    \item[(b)] The additive and multiplicative identities are $(0,0)$ and $(1,1)$.
    \item[(c)] The projections $\pi: R \times R' \to R$ and $\pi': R \times R' \to R'$ defined by $\pi(x,x') = x$ and $\pi'(x,x') = x'$ are ring homomorphisms. The kernels are the ideals $\{0\} \times R'$ and $R \times \{0\}$ of $R \times R'$.
    \item[(d)] The kernel of $\pi'$ is a ring with multiplicative identity $e = (1,0)$. It is not a subring of of $R \times R'$ unless $R'$ is the zero ring. The same holds for the kernel of $\pi$, but the identity is $(0,1)$.
  \end{enumerate}
  \item To see if a ring is isomorphic to a product ring, you must find the elements that would be $(0,1)$ and $(1,0)$. These elements are idempotent.
  \item \textbf{Idempotent}: An element $e$ is \textit{idempotent} if $e^2 = e$
  \item Let $e$ be an idempotent element of the ring $S$.
  \begin{enumerate}
    \item[(a)] The element $e' = 1 - e$ is also idempotent, $e + e' = 1$ and $ee' = 0$
    \item[(b)] The principal ideal $eS$ is a ring with identity element $e$ and multiplication by $e$ defines a ring homomorphism $S \to eS$
    \item[(c)] The ideal $eS$ is not a subring of $S$ unless $e$ is the unit element 1 of $S$ and $e' = 0$
    \item[(d)] The ring $S$ is isomorphic to the product ring $eS \times e'S$
  \end{enumerate}
\end{itemize}

\subsection*{11.7: Fractions}
\begin{itemize}
  \item \textbf{Integral Domain}: A ring $R$ that is not the zero ring, and if $a$ and $b$ are elements of $R$ whose product $ab$ is zero, then $a = 0$ or $b = 0$
  \begin{itemize}
    \item Any subring of a field is a domain, and if $R$ is a domain, the polynomial ring $R[x]$ is also a domain.
  \end{itemize}
  \item \textbf{Zero Divisor}: An element $a$ of a ring that is nonzero and there is another nonzero element $b$ such that $ab = 0$
  \item \textbf{Cancellation Law}: If $ab = ac$ and $a \neq 0$ then $b = c$
  \begin{itemize}
    \item Integral domains satisfy this law
  \end{itemize}
  \item Let $F$ be the set of equivalence classes of fractions of elements of an integral domain $R$
  \begin{enumerate}
    \item[(a)] $F$ is a field, called the \textit{fraction field} of $R$
    \item[(b)] $R$ embeds as a subring of $F$ by the rule $a \leadsto a/1$
    \item[(c)] If $R$ is embedded as a subring of another field $\mathcal{F}$, the rule $a/b = ab^{-1}$ embeds $F$ into $\mathcal{F}$ too (\textit{mapping property})
  \end{enumerate}
  \item \textbf{Mapping Property}: Say the embedding of $R$ into $\mathcal{F}$ is given by the injective ring homomorphism $\phi: R \to \mathcal{F}$. The mapping property states that the rule $\Phi(a/b) = \phi(a)\phi(b)^{-1}$ extends $\phi$ to an injective homomorphism $\Phi: F \to \mathcal{F}$
  \item \textbf{Rational Function}: A fraction of polynomials
  \item \textbf{Field of Rational Function in \textit{x}}: Fractional field of the polynomial ring $K[x]$ where $K$ is a field and coefficients are in $K$. This field is usually denoted $K(x)$:
  \begin{center}
    K(x) = \left\{
      \begin{tabular}{ll}
        equivalence classes of fractions $f/g$, where $f$ and $g$ \\
        are polynomials, and $g$ is not the zero polynomial\\
      \end{tabular}
    \right\}
  \end{center}
\end{itemize}

\subsection*{11.8: Maximal Ideals}
\begin{itemize}
  \item Let $\phi: R \to F$ where $R$ is a ring and $F$ is a field. $F$ has two ideals, the zero ideal (0) and the unit ideal (1). The inverse image of the zero ideal is the kernel $I$ of $\phi$ and the inverse image of the unit ideal is the unit ideal of $R$. Based on the Correspondence Theorem, we know that the only ideals of $R$ that contain $I$ are $I$ and $R$. This means that $I$ is called the \textit{maximal ideal}.
  \item \textbf{Maximal Ideal}: A \textit{maximal ideal} $M$ of a ring $R$ is an ideal that isn't equal to $R$ and isn't contained in any ideal other than $M$ and $R$. If an ideal $I$ contains $M$, then $I = M \lor I = R$.
  \begin{itemize}
    \item A maximal ideal must be a proper ideal
  \end{itemize}
  \item Let $\phi: R \to R'$ be a surjective ring homomorphism with kernel $I$
  \begin{enumerate}
    \item[(a)] The image $R'$ is a field iff $I$ is a maximal ideal
    \item[(b)] An ideal $I$ of a ring $R$ is maximal iff $\bar{R} = R/I$ is a field
    \item[(c)] The zero ideal of a ring $R$ is a maximal iff $R$ is a field
  \end{enumerate}
  \item The maximal ideals of the ring $\mathbb{Z}$ of integers are the principal ideals generated by prime numbers
  \item \textbf{Irreducible polynomial}: A polynomial with coefficients in a field that is not constant and not the product of two polynomials (both of which are not constant)
  \item Let $F$ be a field
  \begin{enumerate}
    \item[(a)] The maximal ideals of $F[x]$ are the principal ideals generated by the monic irreducible polynomials
    \item[(b)] Let $\phi: F[x] \to R'$ be a homomorphism to an integral domain $R'$, and let $P$ be the kernel of $\phi$. Either $P$ is a maximal ideal, or $P = (0)$.
  \end{enumerate}
  \item \textbf{Hilbert's Nullstellensatz}: The maximal ideals of the polynomial ring $\mathbb{C}[x_1, ..., x_n]$ are in bijective correspondence with points of complex $n$-dimensional space. A point $a = (a_1,...,a_n)$ of $\mathbb{C}^n$ corresponds to the kernel $M_a$ of the substitution map $s_a: \mathbb{C}[x] \to \mathbb{C}$ that sends $x \leadsto a$. It is the principal ideal generated by the linear polynomial $x - a$.
  \item Let $R$ be a ring that contains the complex number $\mathbb{C}$ as a subring.
  \begin{enumerate}
    \item[(a)] The laws of composition on $R$ can be used to make $R$ into a complex vector space.
    \item[(b)] As a vector space, the field $\mathcal{F} = \mathbb{C}[x_1,...,x_n]/M$ is spanned by a countable set of elements.
    \item[(c)] Let $V$ be a vector space over a field, and suppose that $V$ is spanned by a countable set of vectors.
    \item[(d)] When $\mathbb{C}(x)$ is made into a vector space over $\mathbb{C}$, the uncountable set of rational functions $(x - \alpha)^{-1}$ with $\alpha$ in $\mathbb{C}$ is independent.
  \end{enumerate}
\end{itemize}

\subsection*{11.9: Algebraic Geometry}
\begin{itemize}
  \item \textbf{Zero}: A point $(a_1,...,a_n)$ of $\mathbb{C}^n$ is called a \textit{zero} of a polynomial $f(x_1,...x_n)$ of $n$ variables if $f(a_1,...,a_n) = 0$. We say that $f$ \textit{vanishes} at that point.
  \begin{itemize}
    \item \textbf{Common Zeros}: The \textit{common zeros} of a set $\{f_1,...,f_r\}$ of polynomials are the points of $\mathbb{C}^n$ at which all of them vanish (the solutions of the system of equations $f_1 = \cdots = f_n = 0$)
    \item \textbf{(Algebraic) Variety}: A subset $V$ of complex $n$-space $\mathbb{C}^n$ that is the set of common zeros of a finite number of polynomials in $n$ variables
  \end{itemize}
\end{itemize}

\newpage
\section*{Chapter 11 Exercises}
\textbf{Problem 11.1.1}: Prove that $7 + \sqrt[3]{2}$ and $\sqrt{3} + \sqrt{-5}$ are algebraic numbers
\begin{proof}
We need to show that they are roots of a nonzero polynomial with integer coefficients. We can show that $(7 + \sqrt[3]{2})^3 - 21(7 + \sqrt[3]{2})^2 + 147(7 + \sqrt[3]{2}) - 345 = 0$. This means it can be represented as the root of a polynomial, namely, $x^3 - 21x^2 + 147x - 345$. For $\sqrt{3} + \sqrt{-5}$, let $x = \sqrt{3} + \sqrt{-5}$.
\begin{align*}
  x^2 &= (\sqrt{3} + \sqrt{-5})(\sqrt{3} + \sqrt{-5}) \\
  x^2 &= 3 + 2\sqrt{-15} - 5 \\
  x^2 &= 2\sqrt{-15} - 2 \\
  x^2 + 2 &= 2\sqrt{-15} \\
  (x^2 + 2)^2 &= -60 \\
  (x^2 + 2)^2 + 60 &= 0
\end{align*}
This means that $\sqrt{3} + \sqrt{-5}$ can be represented as the root of a polynomial.
\end{proof}

\textbf{Problem 11.1.3}: Let $\mathbb{Q}[\alpha, \beta]$ denote the smallest subring of $\mathbb{C}$ containing the rational numbers $\mathbb{Q}$ and the elements $\alpha = \sqrt{2}$ and $\beta = \sqrt{2}$. Let $\gamma = \alpha + \beta$. Is $\mathbb{Q}[\alpha, \beta] = \mathbb{Q}[\gamma]$? Is $\mathbb{Z}[\alpha, \beta] = \mathbb{Z}[\gamma]$?
\begin{proof}
$\mathbb{Q}[\alpha, \beta] = \mathbb{Q}[\gamma]$. To show this, we need to show that $\mathbb{Q}[\alpha, \beta] \subseteq \mathbb{Q}[\gamma]$ and $\mathbb{Q}[\gamma] \subseteq \mathbb{Q}[\alpha, \beta]$. By definition of a subring, we know that $(\alpha + \beta) \in \mathbb{Q}[\alpha, \beta]$, so we know that $\mathbb{Q}[\gamma] \subseteq \mathbb{Q}[\alpha, \beta]$. Now we need to show that $\alpha$ and $\beta$ are in $\mathbb{Q}[\gamma]$. Since $\gamma = \alpha + \beta$, we know that $\gamma^3 = 11\alpha + 9\beta$ is also in $\mathbb{Q}[\gamma]$.
\begin{align*}
  \gamma^3 - 9\gamma &= 2\alpha \\
  \frac{1}{2}\big[\gamma^3 - 9\gamma\big] &= \alpha \\
\end{align*}
Since $\frac{1}{2}$ is in $\mathbb{Q}$, we know that $\alpha$ is in $\mathbb{Q}[\gamma]$. A similar argument can be made to show that $\beta$ is in $\mathbb{Q}[\gamma]$. Since we have shown that $\alpha$ and $\beta$ are in $\mathbb{Q}[\gamma]$, we know that $\mathbb{Q}[\gamma] \subseteq \mathbb{Q}[\alpha, \beta]$, $\therefore \mathbb{Q}[\alpha, \beta] = \mathbb{Q}[\gamma]$. \\

$\mathbb{Z}[\alpha, \beta] \neq \mathbb{Z}[\gamma]$, but I don't know how to prove it. My intuition is that the difference between the two coefficients in a $x\alpha + y\beta$ term will never be 1, and we aren't able to use fractions, so we'll never be able to get $\alpha$ or $\beta$ on its own.
\end{proof}

\textbf{Problem 11.1.6}: Decide whether or not $S$ is a subring of $R$, when
\begin{enumerate}
  \item[(a)] $S$ is the set of all rational numbers $a/b$, where $b$ is not divisible by 2, and $R$ = $\mathbb{Q}$
  \begin{proof}
    $S$ is closed under multiplication because if we multiply $\frac{a}{b}\frac{c}{d}$, we get $\frac{ac}{bd}$, and we know there is no 3 to factor out of the denominator by definition. $S$ is closed under addition because $\frac{a}{b} + \frac{c}{d} = \frac{ad + cb}{bd}$, where again, a 3 cannot be factored out of the denominator. A similar argument can be made for subtraction (since the denominator is the same). $S$ obviously contains 1 ($\frac{1}{1}$), so $S$ is a subring of $\mathbb{Q}$.
  \end{proof}
  \item[(b)] $S$ is the set of functions which are linear combinations with integer coefficients of the functions {1, cos $nt$, sin $nt$}, n $\in \mathbb{Z}$ and $R$ is the set of all real valued functions of $t$.
  \begin{proof}
    $S$ is not a subring of $R$ because it is not closed under multiplication. $sin(x)cos(x) = \frac{1}{2}sin(2x)$. Since you can't write this as a linear combination of the other functions, you know that it is not in $R$ and $S$ is not closed under multiplication.
  \end{proof}
\end{enumerate}

\textbf{Problem 11.1.7}
\begin{enumerate}
  \item[(a)]
  \begin{proof}
  \end{proof}
  \item[(b)]
  \begin{proof}
  \end{proof}
\end{enumerate}

\textbf{Problem 11.1.8}
\begin{proof}

\end{proof}

\textbf{Problem 11.2.2}
\begin{proof}

\end{proof}

\textbf{Problem 11.3.1}
\begin{proof}

\end{proof}

\textbf{Problem 11.3.2}
\begin{proof}

\end{proof}

\textbf{Problem 11.3.3}
\begin{proof}

\end{proof}

\textbf{Problem 11.3.5}
\begin{proof}

\end{proof}

\textbf{Problem 11.3.6}
\begin{proof}

\end{proof}

\textbf{Problem 11.3.7}
\begin{proof}

\end{proof}

\textbf{Problem 11.3.8}
\begin{proof}

\end{proof}

\textbf{Problem 11.3.9}
\begin{proof}

\end{proof}

\textbf{Problem 11.4.1}
\begin{proof}

\end{proof}

\textbf{Problem 11.4.2}
\begin{proof}

\end{proof}

\textbf{Problem 11.5.1}
\begin{proof}

\end{proof}

\textbf{Problem 11.5.2}
\begin{proof}

\end{proof}

\textbf{Problem 11.5.3}
\begin{proof}

\end{proof}
\textbf{Problem 11.5.6}
\begin{proof}

\end{proof}
\textbf{Problem 11.5.7}
\begin{proof}

\end{proof}
\textbf{Problem 11.6.2}
\begin{proof}

\end{proof}
\textbf{Problem 11.6.2}
\begin{proof}

\end{proof}
\textbf{Problem 11.6.8}
\begin{proof}

\end{proof}

\textbf{Problem 11.7.1}
\begin{proof}
for any element R, you can construct a map
show there are inverses
\end{proof}

\textbf{Problem 11.7.2}
\begin{proof}

\end{proof}

\textbf{Problem 11.7.5}
\begin{proof}

\end{proof}

\textbf{Problem 11.8.1}: Which principal ideals in $\mathbb{Z}[x]$ are maximal ideals?
\begin{proof}
$\mathbb{Z}[x]$ contains all polynomials of the form $a_nx^n + \cdots + a_1x + a_0$ where $a_i$ are in $\mathbb{Z}$
\end{proof}

\textbf{Problem 11.8.2}
\begin{proof}

\end{proof}
\textbf{Problem 11.8.4}
\begin{proof}

\end{proof}
\textbf{Problem 11.9.1}
\begin{proof}

\end{proof}
\textbf{Problem 11.9.2}
\begin{proof}

\end{proof}
\textbf{Problem 11.9.3}
\begin{proof}

\end{proof}
\textbf{Problem 11.9.4}
\begin{proof}

\end{proof}
\textbf{Problem 11.9.5}
\begin{proof}

\end{proof}
\textbf{Problem 11.9.6}
\begin{proof}

\end{proof}
\textbf{Problem 11.9.9}
\begin{proof}

\end{proof}
\textbf{Problem 11.9.10}
\begin{proof}

\end{proof}
\textbf{Problem 11.9.11}
\begin{proof}

\end{proof}
\textbf{Problem 11.9.12}
\begin{proof}

\end{proof}
\textbf{Problem 11.9.12}
\begin{proof}

\end{proof}
\textbf{Problem 11.M.1}
\begin{proof}

\end{proof}
\textbf{Problem 11.M.2}
\begin{proof}

\end{proof}
\textbf{Problem 11.M.3}
\begin{proof}

\end{proof}
\textbf{Problem 11.M.5}
\begin{proof}

\end{proof}
\textbf{Problem 11.M.6}
\begin{proof}

\end{proof}

\end{document}
