\documentclass[12pt]{article}
\usepackage[utf8]{inputenc}
\usepackage[margin=1in, top=1.5in]{geometry}
\usepackage{fancyhdr, amsthm, amssymb, amsmath}
\setlength\parindent{0pt}

\pagestyle{fancy}
\fancyhf{}
\lhead{Algebra}
\rhead{Notes}
\rfoot{\thepage}

\begin{document}

\section*{Chapter 11: Rings}

\subsection*{11.1: Definition of a Ring}
\begin{itemize}
  \item \textbf{Ring}: A \textit{ring} $R$ is a set with two laws of composition $+$ and $\times$, called addition and multiplication, that satisfy these axioms:
  \begin{enumerate}
    \begin{enumerate}
      \item With the law of composition $+$, $R$ is an abelian group that we denote by $R^+$; its identity is denoted by 0.
      \item Multiplication is commutative and associative, and has an identity denoted by 1.
      \item \textit{Distributive law}: For all $a, b$ and $c$ in $R$, $(a + b)c = ac + bc$.
    \end{enumerate}
  \end{enumerate}
  \begin{itemize}
    \item \textbf{Subring}: Subset which is closed under addition, subtraction, multiplication and which contains 1.
    \item \textbf{Non-commutative Ring}: Satisfies all of the above axioms, except for the commutative law for multiplication.
  \end{itemize}
  \item \textbf{Gauss integers}: The complex numbers of the form $a + bi$ where $a$ and $b$ are integers form a subring of $\mathbb{C}$ that we denote by $\mathbb{Z}$[i] = \{$a + bi \mid b, b \in \mathbb{Z}$\}. Its elements are points of a square lattice in the complex plane.
  \begin{itemize}
    \item \textbf{$\mathbb{Z}$[$\alpha$] subring}: Contains every complex number $\beta$ = $a_n\alpha^n + ... + a_1\alpha + a_0$ where $a_i$ are in $\mathbb{Z}$ and $\alpha$ is a complex number.
    \begin{itemize}
      \item Analogous to the ring of Gauss integers.
      \item Subring generated by $\alpha$
      \item Usually not represented as a lattice in the complex plane
    \end{itemize}
  \end{itemize}
  \item A complex number $\alpha$ is \textbf{algebraic} if it is a root of a (nonzero) polynomial with integer coefficients (i.e. if some expression of the form $a_n\alpha^n + ... + a_1\alpha + a_0$ evaluates to 0)
  \begin{itemize}
    \item When $\alpha$ is algebraic there will be many polynomial expressions that represent the same complex number.
  \end{itemize}
  \item If there is no polynomial with integer coefficients having $\alpha$ as a root, $\alpha$ is \textbf{transcendental}
  \begin{itemize}
    \item When $\alpha$ is transcendental, two distinct polynomial expressions represent distinct complex numbers, and the elements of the ring $\mathbb{Z}$[$\alpha$] correspond bijectively to polynomials $p(x)$ with integer coefficients.
  \end{itemize}
  \item A polynomial in $x$ with coefficients in a ring $R$ is an expression of the form
  \begin{center}
    $a_nx^n + ... + a_1x + a_0$
  \end{center}
  with $a_i$ in $R$.
  \item \textbf{Zero Ring}: A ring containing only  the element 0.
  \begin{itemize}
    \item A ring $R$ in which the elements 1 and 0 are equal is the zero ring.
  \end{itemize}
  \item \textbf{Unit}: A \textit{unit} of a ring is an element that has a multiplicative inverse (if it exists, it is unique)
  \begin{itemize}
    \item Units in the ring of integers are 1 and -1
    \item Units in the ring of Gauss integers are $\pm$1 and $\pm$i
    \item Units in the ring $\mathbb{R}$[$x$] of real polynomials are the nonzero constant polynomials
    \item The identity element 1 of a ring is always a unit
  \end{itemize}
\end{itemize}

\subsection*{11.2: Polynomial Rings}
\begin{itemize}
  \item \textbf{Formal Polynomial}: A polynomial with coefficients in a ring $R$ is a (finite) linear combination of powers of the variable: $f(x) = a_nx^n + a_{n-1}x^{n-1} + ... + a_1x + a_0$ where the coefficients $a_i$ are elements of $R$.
  \begin{itemize}
    \item The set of polynomials with coefficients in a ring $R$ will be denoted $R$[$x$]
    \item Thus $\mathbb{Z}$[$x$] is the set of \textit{integer polynomials}
  \end{itemize}
  \item The \textit{monomials} $x^i$ are considered independent, so if $\exists$ another polynomial with coefficients in $R$, then $f(x) = g(x)$ only if $a_i = b_i$ for all $i = 0, 1, 2, ...$
  \item \textbf{Degree}: The \textit{degree} of a nonzero polynomial (denoted deg $f$) is the largest integer $n$ such that the coefficient $a_n$ of $x_n$ is not zero
  \begin{itemize}
    \item A polynomial of degree zero is called a \textit{constant} polynomial
    \item The zero polynomial is also a constant polynomial, but its degree will not be defined
  \end{itemize}
  \item \textbf{Leading Coefficient}: The nonzero coefficient of highest degree of a polynomial
  \begin{itemize}
    \item \textbf{Monic Polynomial}: Polynomial with a leading coefficient of 1
  \end{itemize}
  \item A polynomial is determined by its vector of coefficients $a_i$: $a = (a_0, a_1, ...)$ where $a_i$ are elements of $R$, all but a finite number zero.
  \item When $R$ is a field, these infinite vectors form the vector space $Z$ with the infinite basis $e_i$. The vector $e_i$ corresponds to the monomial $x_i$, and the monomials form a basis of the space of all polynomials.
  \item \textbf{Addition of polynomials}: $f(x) + g(x) = (a_0 + b_0) + (a_1 + b_1)x + ...$ where $(a_i + b_i)$ is addition in $R$
  \item \textbf{Multiplication of polynomials}: $f(x)g(x) = (a_0 + a_1x + ...)(b_0 + b_1x + ...)$ where $a_ib_j$ are to be evaluated in the ring $R$.
  \item There is a unique commutative ring structure on the set of polynomials $R[x]$ having these properties:
  \begin{itemize}
    \item Additions of polynomials as defined above
    \item Multiplication of polynomials as defined above
    \item The ring $R$ becomes a subring of $R[x]$ when the elements of $R$ are identifies with the constant polynomials
  \end{itemize}
  \item \textbf{Division with Remainder}: Let $R$ be a ring, $f$ is a monic polynomial, and $g$ is any polynomial, both with coefficients in $R$. There are uniquely determined polynomials $q$ and $r$ in $R[x]$ s.t. $g(x) = f(x)q(x) + r(x)$ where $r$ has degree $\geqslant 0$ and $\leqslant f$
  \begin{itemize}
    \item Division with remainder can be done whenever the leading coefficient of $f$ is a unit
    \item If $g(x)$ is a polynomial in $R[x]$ and $\alpha$ is an element of $R$, the remainder of division of $g(x)$ by $x - \alpha$ is $g(\alpha)$. Thus $x - \alpha$ divides $g$ in $R[x]$ iff $g(\alpha) = 0$
  \end{itemize}
  \item \textbf{Monomial}: a formal product of some variables $x_1, ..., x_n$ of the form
  \begin{center}
    $x_1^{i_1}x_2^{i_2}...x_n^{i_n}$
  \end{center}
  where $i_v$ are non-negative integers.
  \begin{itemize}
    \item \textbf{Degree}: the sum $i_1 + ... + i_n$, sometimes called \textit{total degree}
    \item \textbf{Multi-index}: an $n$-tuple that can be represented with vector notation e.g.  $i = (i_1, ... i_n)$.
    \item A monomial can be written as $x^i$ ($= x_1^{i_1}x_2^{i_2} ... x_n^{i_n}$) using multi-index form
    \item The monomial $x^0$ is denoted by 1
  \end{itemize}
  \item With multi-index notation, a polynomial $f(x) = f(x_1, ..., x_n)$ can be written in exactly one way in the form
  \begin{center}
    $f(x) = \sum\limits_ia_ix^i$
  \end{center}
  where $i$ runs through all multi-indices $(i_1, ..., i_n)$, the coefficients $a_i$ are in $R$ and only finitely many of these coefficients are not 0.
  \item \textbf{Homogeneous Polynomial}: A polynomial in which all monomials with nonzero coefficients have degree $d$
\end{itemize}

\subsection*{11.3: Homomorphisms and Ideals}
\begin{itemize}
  \item \textbf{Ring Homomorphism}: A \textit{ring homomorphism} $\phi: R \to R'$ is a map from one ring to another which is compatible with the laws of composition and which carries the unit element 1 of $R$ to the unit element 1 of $R'$ - a map such that for all $a$ and $b$ in $R$,
  \begin{center}
    $\phi(a + b) = \phi(a) + \phi(b), \quad \phi(ab) = \phi(a)\phi(b), \quad and \quad \phi(1) = 1$
  \end{center}
  \begin{itemize}
    \item The map $\phi: \mathbb{Z} \to \mathbb{F}_p$ that send an integer to its congruence class modulo $p$ is a ring homomorphism.
  \end{itemize}
  \item \textbf{Isomorphism}: An \textit{isomorphism} of rings is a bijective homomorphism, denoted $R \approx R'$
  \item Evaluation of real polynomials at a real number $a$ defines a homomorphism
  \begin{center}
    $\mathbb{R}[x] \to \mathbb{R}$, \quad that sends \quad $p(x) \leadsto p(a)$
  \end{center}
  \item \textbf{Substitution Principle}: Let $\phi: R \to R'$ be a ring homomorphism, and let $R[x]$ be the ring of polynomials with coefficients in $R$.
  \begin{enumerate}
    \begin{enumerate}
      \item Let $\alpha$ be an element of $R'$. There is a unique homomorphism $\Phi: R[x] \to R'$ that agrees with the map $\phi$ on constant polynomials, and that send $x \leadsto a$
      \item Given elements $\alpha_1, ..., \alpha_n$ of $R'$, there is a unique homomorphism $\Phi: R[x_1, ..., x_n] \to R'$, from the polynomial ring in $n$ variables to $R'$, that agrees with $\phi$ on constant polynomials and that send $x_v \leadsto \alpha_v$, for $v = 1, ..., n$.
    \end{enumerate}
  \end{enumerate}
  \item Let $R$ be any ring, and let $P$ be the polynomial ring $R[x]$. One can use the substitution principle to construct an isomorphism
  \begin{center}
    $R[x,y] \to P[y] = (R[x])[y]$
  \end{center}
  This statement is a formalization of the procedure of collecting terms of like degree in $y$ in a polynomial $f(x,y)$. For example:
  \begin{center}
    $x^4y + x^3 - 3x^2y + y^2 + 2 = y^2 + (x^4 - 3x^2)y + (x^3 + 2)$
  \end{center}
  \item Let $x = (x_1, ..., x_m)$ and $y = (y_1, ..., y_n)$ denote sets of variables. There is a unique isomorphism $R[x,y] \to R[x][y]$, which is the identity on $R$ and sends the variables to themselves.
  \item Let $f(x,y)$ and $g(x,y)$ be polynomials in two variables, elements of $R[x,y]$. Suppose that $f$ is a monic polynomial of degree $m$ (grouped by $y$). There are uniquely determined polynomials $q(x,y)$ and $r(x,y)$ such that $g = fq + r$ and $0 \leqslant r(x,y) < m$
  \item There is exactly one homomorphism $\phi: \mathbb{Z} \to R$, defined for $n \geqslant 0$ where $\phi(n) = 1 + ... + 1$ (for $n$ terms) and $\phi(-n) = -\phi(n)$
  \item \textbf{Kernel}: The \textit{kernel} of $\phi$ is the set of elements $R$ that map to zero:
  \begin{center}
    ker$\phi = \{s \in R \mid \phi(s) = 0\}$
  \end{center}
  \begin{itemize}
    \item If $s$ is in $ker\phi$, then for every element $r$ of $R$, $rs$ is in $ker\phi$
  \end{itemize}
  \item \textbf{Ideal}: An \textit{ideal} $I$ of a ring $R$ is a nonempty subset of $R$ with these properties:
  \begin{enumerate}
    \begin{enumerate}
      \item $I$ is closed under addition, and
      \item If $s$ is in $I$ and $r$ is in $R$, then $rs$ is in $I$
    \end{enumerate}
  \end{enumerate}
  \begin{itemize}
    \item \textbf{Principal Ideal}: The ideal formed by multiples of a particular element $a$, also defined as:
    \begin{center}
      $(a) - aR = Ra = \{ra \mid r \in R\}$
    \end{center}
    \item \textbf{Unit Ideal}: The ring $R$ is the principal ideal (1), and is  called the \textit{unit ideal}
    \item \textbf{Zero Ideal}: The principal ideal (0)
    \item \textbf{Proper Ideal}: An ideal that is neither the unit or zero ideal
  \end{itemize}
  \item The kernel of a ring homomorphism is an ideal
  \item An ideal is not a subring unless the ideal $I$ is equal to the whole ring $R$
  \item The ideal \textit{generated by a set of elements} $\{a_1, ..., a_n\}$ of a ring $R$ is the smallest ideal that contains those elements. This ideal is often denoted as $(a_1, ..., a_n)$:
  \begin{center}
    $(a_1, ..., a_n) = \{r_1a_1 + ... + r_na_n \mid r_i \in R\}$
  \end{center}
  \item The only ideals of a field are the zero ideal and the unit ideal
  \item A ring that has exactly two ideals is a field
  \item Every homomorphism $\phi: F \to R$ from a field $F$ to a nonzero ring $R$ is injective
  \item The ideals in the ring of integers are the subgroups of $\mathbb{Z}^+$, and they are principal ideals
  \item Every ideal in the ring $F[x]$ of polynomials in one variable $x$ over a field $F$ is a principal ideal. A nonzero ideal $I$ in $F[x]$ is generated by the unique monic polynomial of lower degree that it contains.
  \item Let $f$ be a monic integer polynomial, and let $g$ be another integer polynomial. If $f\mid g$ in $\mathbb{Q}[x]$, $f\mid g$ in $\mathbb{Z}[x]$
  \item \textbf{Greatest Common Divisor}: Let $R$ denote the polynomial ring $F[x]$ in one variable over a field $F$, and let $f$ and $g$ be elements of $R$, not both zero. Their \textit{greatest common divisor} $d(x)$ is the unique monic polynomials that generates the ideal $(f,g)$. It has these properties:
  \begin{enumerate}
    \begin{enumerate}
      \item $Rd = Rf + Rg$
      \item $d$ divides $f$ and $g$
      \item If a polynomial $e = e(x)$ divides both $f$ and $g$, it also divides $d$
      \item There are polynomials $p$ and $q$ such that $d = pf + qg$
    \end{enumerate}
  \end{enumerate}
  \item \textbf{Characteristic}: The non-negative integer $n$ that generates the kernel of the homomorphism $\phi: \mathbb{Z} \to R$
  \begin{enumerate}
    \item If $n = 0$, this means that no positive multiple of 1 in $R$ is equal to zero. Otherwise $n$ is the smallest positive integer s.t. "$n$ times 1" is zero in R
  \end{enumerate}
\end{itemize}

\subsection*{11.4: Quotient Rings}
\begin{itemize}

\end{itemize}

\section*{Chapter 11 Exercises}
\textbf{Problem 11.1.1}: Prove that $7 + \sqrt[3]{2}$ and $\sqrt{3} + \sqrt{-5}$ are algebraic numbers
\begin{proof}
We need to show that they are roots of a nonzero polynomial with integer coefficients. We can show that $(7 + \sqrt[3]{2})^3 - 21(7 + \sqrt[3]{2})^2 + 147(7 + \sqrt[3]{2}) - 345 = 0$. This means it can be represented as the root of a polynomial, namely, $x^3 - 21x^2 + 147x - 345$. For $\sqrt{3} + \sqrt{-5}$, let $x = \sqrt{3} + \sqrt{-5}$.
\begin{align*}
  x^2 &= (\sqrt{3} + \sqrt{-5})(\sqrt{3} + \sqrt{-5}) \\
  x^2 &= 3 + 2\sqrt{-15} - 5 \\
  x^2 &= 2\sqrt{-15} - 2 \\
  x^2 + 2 &= 2\sqrt{-15} \\
  (x^2 + 2)^2 &= -60 \\
  (x^2 + 2)^2 + 60 &= 0
\end{align*}
This means that $\sqrt{3} + \sqrt{-5}$ can be represented as the root of a polynomial.
\end{proof}

\textbf{Problem 11.1.3}: Let $\mathbb{Q}[\alpha, \beta]$ denote the smallest subring of $\mathbb{C}$ containing the rational numbers $\mathbb{Q}$ and the elements $\alpha = \sqrt{2}$ and $\beta = \sqrt{2}$. Let $\gamma = \alpha + \beta$. Is $\mathbb{Q}[\alpha, \beta] = \mathbb{Q}[\gamma]$? Is $\mathbb{Z}[\alpha, \beta] = \mathbb{Z}[\gamma]$?
\begin{proof}
$\mathbb{Q}[\alpha, \beta] = \mathbb{Q}[\gamma]$. To show this, we need to show that $\mathbb{Q}[\alpha, \beta] \subseteq \mathbb{Q}[\gamma]$ and $\mathbb{Q}[\gamma] \subseteq \mathbb{Q}[\alpha, \beta]$. By definition of a subring, we know that $(\alpha + \beta) \in \mathbb{Q}[\alpha, \beta]$, so we know that $\mathbb{Q}[\gamma] \subseteq \mathbb{Q}[\alpha, \beta]$. Now we need to show that $\alpha$ and $\beta$ are in $\mathbb{Q}[\gamma]$. Since $\gamma = \alpha + \beta$, we know that $\gamma^3 = 11\alpha + 9\beta$ is also in $\mathbb{Q}[\gamma]$.
\begin{align*}
  \gamma^3 - 9\gamma &= 2\alpha \\
  \frac{1}{2}\big[\gamma^3 - 9\gamma\big] &= \alpha \\
\end{align*}
Since $\frac{1}{2}$ is in $\mathbb{Q}$, we know that $\alpha$ is in $\mathbb{Q}[\gamma]$. A similar argument can be made to show that $\beta$ is in $\mathbb{Q}[\gamma]$. Since we have shown that $\alpha$ and $\beta$ are in $\mathbb{Q}[\gamma]$, we know that $\mathbb{Q}[\gamma] \subseteq \mathbb{Q}[\alpha, \beta]$, $\therefore \mathbb{Q}[\alpha, \beta] = \mathbb{Q}[\gamma]$. \\

$\mathbb{Z}[\alpha, \beta] \neq \mathbb{Z}[\gamma]$, but I don't know how to prove it. My intuition is that the difference between the two coefficients in a $x\alpha + y\beta$ term will never be 1, and we aren't able to use fractions, so we'll never be able to get $\alpha$ or $\beta$ on its own.
\end{proof}

\textbf{Problem 11.1.6}: Decide whether or not $S$ is a subring of $R$, when
\begin{enumerate}
  \item[(a)] $S$ is the set of all rational numbers $a/b$, where $b$ is not divisible by 2, and $R$ = $\mathbb{Q}$
  \begin{proof}
    $S$ is closed under multiplication because if we multiply $\frac{a}{b}\frac{c}{d}$, we get $\frac{ac}{bd}$, and we know there is no 3 to factor out of the denominator by definition. $S$ is closed under addition because $\frac{a}{b} + \frac{c}{d} = \frac{ad + cb}{bd}$, where again, a 3 cannot be factored out of the denominator. A similar argument can be made for subtraction (since the denominator is the same). $S$ obviously contains 1 ($\frac{1}{1}$), so $S$ is a subring of $\mathbb{Q}$.
  \end{proof}
  \item[(b)] $S$ is the set of functions which are linear combinations with integer coefficients of the functions {1, cos $nt$, sin $nt$}, n $\in \mathbb{Z}$ and $R$ is the set of all real valued functions of $t$.
  \begin{proof}
    $S$ is not a subring of $R$ because it is not closed under multiplication. $sin(x)cos(x) = \frac{1}{2}sin(2x)$. Since you can't write this as a linear combination of the other functions, you know that it is not in $R$ and $S$ is not closed under multiplication.
  \end{proof}
\end{enumerate}

\textbf{Problem 11.1.7}
\begin{enumerate}
  \begin{enumerate}
    \item
    \begin{proof}
    \end{proof}
    \item
    \begin{proof}
    \end{proof}
  \end{enumerate}
\end{enumerate}

\textbf{Problem 11.1.8}
\begin{proof}

\end{proof}

\textbf{Problem 11.2.2}
\begin{proof}

\end{proof}

\textbf{Problem 11.3.1}
\begin{proof}

\end{proof}

\textbf{Problem 11.3.2}
\begin{proof}

\end{proof}

\textbf{Problem 11.3.3}
\begin{proof}

\end{proof}

\textbf{Problem 11.3.5}
\begin{proof}

\end{proof}

\textbf{Problem 11.3.6}
\begin{proof}

\end{proof}

\textbf{Problem 11.3.7}
\begin{proof}

\end{proof}

\textbf{Problem 11.3.8}
\begin{proof}

\end{proof}

\textbf{Problem 11.3.9}
\begin{proof}

\end{proof}

\textbf{Problem 11.4.1}
\begin{proof}

\end{proof}

\textbf{Problem 11.4.2}
\begin{proof}

\end{proof}

\textbf{Problem 11.5.1}
\begin{proof}

\end{proof}

\textbf{Problem 11.5.2}
\begin{proof}

\end{proof}

\textbf{Problem 11.5.3}
\begin{proof}

\end{proof}
\textbf{Problem 11.5.6}
\begin{proof}

\end{proof}
\textbf{Problem 11.5.7}
\begin{proof}

\end{proof}
\textbf{Problem 11.6.2}
\begin{proof}

\end{proof}
\textbf{Problem 11.6.2}
\begin{proof}

\end{proof}
\textbf{Problem 11.6.8}
\begin{proof}

\end{proof}
\textbf{Problem 11.7.1}
\begin{proof}

\end{proof}
\textbf{Problem 11.7.2}
\begin{proof}

\end{proof}
\textbf{Problem 11.7.5}
\begin{proof}

\end{proof}
\textbf{Problem 11.8.1}
\begin{proof}

\end{proof}
\textbf{Problem 11.8.2}
\begin{proof}

\end{proof}
\textbf{Problem 11.8.4}
\begin{proof}

\end{proof}
\textbf{Problem 11.9.1}
\begin{proof}

\end{proof}
\textbf{Problem 11.9.2}
\begin{proof}

\end{proof}
\textbf{Problem 11.9.3}
\begin{proof}

\end{proof}
\textbf{Problem 11.9.4}
\begin{proof}

\end{proof}
\textbf{Problem 11.9.5}
\begin{proof}

\end{proof}
\textbf{Problem 11.9.6}
\begin{proof}

\end{proof}
\textbf{Problem 11.9.9}
\begin{proof}

\end{proof}
\textbf{Problem 11.9.10}
\begin{proof}

\end{proof}
\textbf{Problem 11.9.11}
\begin{proof}

\end{proof}
\textbf{Problem 11.9.12}
\begin{proof}

\end{proof}
\textbf{Problem 11.9.12}
\begin{proof}

\end{proof}
\textbf{Problem 11.M.1}
\begin{proof}

\end{proof}
\textbf{Problem 11.M.2}
\begin{proof}

\end{proof}
\textbf{Problem 11.M.3}
\begin{proof}

\end{proof}
\textbf{Problem 11.M.5}
\begin{proof}

\end{proof}
\textbf{Problem 11.M.6}
\begin{proof}

\end{proof}

\end{document}
