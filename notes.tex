\documentclass[12pt]{article}
\usepackage[utf8]{inputenc}
\usepackage[margin=1in, top=1.5in]{geometry}
\usepackage{fancyhdr}
\usepackage{amsthm}
\usepackage{amssymb}

\pagestyle{fancy}
\fancyhf{}
\lhead{Algebra}
\rhead{Notes}

\begin{document}

\section*{Chaper 11: Rings}

\subsection*{11.1: Definition of a Ring}
\begin{itemize}
  \item \textbf{Ring}: A \textit{ring} $R$ is a set with two laws of composition $+$ and $\times$, called addition and multiplication, that satisfy these axioms:
  \begin{enumerate}
    \begin{enumerate}
      \item With the law of composition $+$, $R$ is an abelian group that we denote by $R^+$; its identity is denoted by 0.
      \item Multiplication is commutative and associative, and has an identity denoted by 1.
      \item \textit{Distributive law}: For all $a, b$ and $c$ in $R$, $(a + b)c = ac + bc$.
    \end{enumerate}
  \end{enumerate}
  \begin{itemize}
    \item \textbf{Subring}: Subset which is closed under addition, subtraction, multiplication and which contains 1.
    \item \textbf{Noncommutative Ring}: Satisfies al of the above axioms, except for the commutative law for multiplication.
  \end{itemize}
  \item \textbf{Gauss integers}: The complex numbers of the form $\textit{a} + \textit{bi}$ where \textit{a} and \textit{b} are integers form a subring of $\mathbb{C}$ that we denote by $\mathbb{Z}$[i] = \{$\textit{a} + \textit{bi} \mid \textit{a}, \textit{b} \in \mathbb{Z}$\}. Its elements are points of a square lattice in the complex plane.
  \begin{itemize}
    \item \textbf{$\mathbb{Z}$[$\alpha$] subring}: Ccontains every complex number $\beta$ = $a_n\alpha^n + ... + a_1\alpha + a_0$ where $a_i$ are in $\mathbb{Z}$ and $\alpha$ is a complex number.
    \begin{itemize}
      \item Analogous to the ring of Gauss integers.
      \item Subring generated by $\alpha$
      \item Usually not represented as a lattice in the complex plane
    \end{itemize}
  \end{itemize}
  \item A complex number $\alpha$ is \textbf{algebraic} if it is a root of a (nonzero) polynomial with integer coefficients (i.e. if some expression of the form $a_n\alpha^n + ... + a_1\alpha + a_0$ evaluates to 0)
  \begin{itemize}
    \item When $\alpha$ is algebraic there will be many polynomial expressions that represent the same complex number.
  \end{itemize}
  \item If there is no polynomial with integer coefficients having $\alpha$ as a root, $\alpha$ is \textbf{transcendental}
  \begin{itemize}
    \item When $\alpha$ is transcendental, two distinct polynomial expressions represent distinct complex numbers, and the elements of the ring $\mathbb{Z}$[$\alpha$] correspond bijectively to polynomials $p(x)$ with integer coefficients.
  \end{itemize}
  \item A polynomial in $x$ with coefficients in a ring $R$ is an expression of the form $a_nx^n + ... + a_1x + a_0$ with $a_i$ in $R$.
  \item \textbf{Zero Ring}: A ring containing only  the element 0.
  \begin{itemize}
    \item A ring $R$ in which the elements 1 and 0 are equal is the zero ring.
  \end{itemize}
  \item \textbf{Unit}: A \textit{unit} of a ring is an element that has a multiplicative inverse (if it exists, it is unique)
  \begin{itemize}
    \item Units in the ring of integers are 1 and -1
    \item Units in the ring of Gauss integers are $\pm$1 and $\pm$i
    \item Units in the ring $\mathbb{R}$[$x$] of real polynomials are the nonzero constant polynomials
    \item The identity element 1 of a ring is always a unit
  \end{itemize}
\end{itemize}

\subsection*{11.2: Polynomial Rings}
\begin{itemize}
  \item \textbf{Formal Polynomial}: A polynomial with coefficients in a ring $R$ is a (finite) linear combination of powers of the variable: $f(x) = a_nx^n + a_{n-1}x^{n-1} + ... + a_1x + a_0$ where the coefficients $a_i$ are elements of $R$.
\end{itemize}

\iffalse
\subsection*{Problem 1.1}
\textbf{Prove that 7 + $\sqrt[3]{2}$ and $\sqrt{3}$ + $\sqrt{-5}$ are algeabraic numbers.}
\begin{proof}
Since p is not in M, it is not between or equal to a and b, and therefore any interval that contains p is not within M.
\end{proof}
\fi


\end{document}
