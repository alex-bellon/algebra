\documentclass[12pt]{article}
\usepackage[utf8]{inputenc}
\usepackage[margin=1in, top=1.5in]{geometry}
\usepackage{fancyhdr, amsthm, amssymb}

\pagestyle{fancy}
\fancyhf{}
\lhead{Algebra}
\rhead{Notes}
\rfoot{\thepage}

\begin{document}

\section*{Chaper 11: Rings}

\subsection*{11.1: Definition of a Ring}
\begin{itemize}
  \item \textbf{Ring}: A \textit{ring} $R$ is a set with two laws of composition $+$ and $\times$, called addition and multiplication, that satisfy these axioms:
  \begin{enumerate}
    \begin{enumerate}
      \item With the law of composition $+$, $R$ is an abelian group that we denote by $R^+$; its identity is denoted by 0.
      \item Multiplication is commutative and associative, and has an identity denoted by 1.
      \item \textit{Distributive law}: For all $a, b$ and $c$ in $R$, $(a + b)c = ac + bc$.
    \end{enumerate}
  \end{enumerate}
  \begin{itemize}
    \item \textbf{Subring}: Subset which is closed under addition, subtraction, multiplication and which contains 1.
    \item \textbf{Noncommutative Ring}: Satisfies al of the above axioms, except for the commutative law for multiplication.
  \end{itemize}
  \item \textbf{Gauss integers}: The complex numbers of the form $a + bi$ where $a$ and $b$ are integers form a subring of $\mathbb{C}$ that we denote by $\mathbb{Z}$[i] = \{$a + bi \mid b, b \in \mathbb{Z}$\}. Its elements are points of a square lattice in the complex plane.
  \begin{itemize}
    \item \textbf{$\mathbb{Z}$[$\alpha$] subring}: Ccontains every complex number $\beta$ = $a_n\alpha^n + ... + a_1\alpha + a_0$ where $a_i$ are in $\mathbb{Z}$ and $\alpha$ is a complex number.
    \begin{itemize}
      \item Analogous to the ring of Gauss integers.
      \item Subring generated by $\alpha$
      \item Usually not represented as a lattice in the complex plane
    \end{itemize}
  \end{itemize}
  \item A complex number $\alpha$ is \textbf{algebraic} if it is a root of a (nonzero) polynomial with integer coefficients (i.e. if some expression of the form $a_n\alpha^n + ... + a_1\alpha + a_0$ evaluates to 0)
  \begin{itemize}
    \item When $\alpha$ is algebraic there will be many polynomial expressions that represent the same complex number.
  \end{itemize}
  \item If there is no polynomial with integer coefficients having $\alpha$ as a root, $\alpha$ is \textbf{transcendental}
  \begin{itemize}
    \item When $\alpha$ is transcendental, two distinct polynomial expressions represent distinct complex numbers, and the elements of the ring $\mathbb{Z}$[$\alpha$] correspond bijectively to polynomials $p(x)$ with integer coefficients.
  \end{itemize}
  \item A polynomial in $x$ with coefficients in a ring $R$ is an expression of the form
  \begin{center}
    $a_nx^n + ... + a_1x + a_0$
  \end{center}
  with $a_i$ in $R$.
  \item \textbf{Zero Ring}: A ring containing only  the element 0.
  \begin{itemize}
    \item A ring $R$ in which the elements 1 and 0 are equal is the zero ring.
  \end{itemize}
  \item \textbf{Unit}: A \textit{unit} of a ring is an element that has a multiplicative inverse (if it exists, it is unique)
  \begin{itemize}
    \item Units in the ring of integers are 1 and -1
    \item Units in the ring of Gauss integers are $\pm$1 and $\pm$i
    \item Units in the ring $\mathbb{R}$[$x$] of real polynomials are the nonzero constant polynomials
    \item The identity element 1 of a ring is always a unit
  \end{itemize}
\end{itemize}

\subsection*{11.2: Polynomial Rings}
\begin{itemize}
  \item \textbf{Formal Polynomial}: A polynomial with coefficients in a ring $R$ is a (finite) linear combination of powers of the variable: $f(x) = a_nx^n + a_{n-1}x^{n-1} + ... + a_1x + a_0$ where the coefficients $a_i$ are elements of $R$.
  \begin{itemize}
    \item The set of polynomials with coefficients in a ring $R$ will be denoted $R$[$x$]
    \item Thus $\mathbb{Z}$[$x$] is the set of \textit{integer polynomials}
  \end{itemize}
  \item The \textit{monomials} $x^i$ are considered independent, so if $\exists$ another polynomial with coefficients in $R$, then $f(x) = g(x)$ only if $a_i = b_i$ for all $i = 0, 1, 2, ...$
  \item \textbf{Degree}: The \textit{degree} of a nonzero polynomial (denoted deg $f$) is the largest integer $n$ such that the coefficient $a_n$ of $x_n$ is not zero
  \begin{itemize}
    \item A polynomial of degree zero is called a \textit{constant} polynomial
    \item The zero polynomial is also a constant polynomial, but its degree will not be defined
  \end{itemize}
  \item \textbf{Leading Coefficient}: The nonzero coefficientof highest degree of a polynomial
  \begin{itemize}
    \item \textbf{Monic Polynomial}: Polynomial with a leading coefficient of 1
  \end{itemize}
  \item A polynomial is determined by its vector of coefficients $a_i$: $a = (a_0, a_1, ...)$ where $a_i$ are elements of $R$, all but a finite number zero.
  \item When $R$ is a field, these infinite vectors form the vector space $Z$ with the infinite basis $e_i$. The vector $e_i$ corresponds to the monomial $x_i$, and the monomials form a basis of the space of all polynomials.
  \item \textbf{Addition of polynomials}: $f(x) + g(x) = (a_0 + b_0) + (a_1 + b_1)x + ...$ where $(a_i + b_i)$ is addition in $R$
  \item \textbf{Multiplication of polynomials}: $f(x)g(x) = (a_0 + a_1x + ...)(b_0 + b_1x + ...)$ where $a_ib_j$ are to be evaluated in the ring $R$.
  \item There is a unique commutative ring structure on the set of polynomials $R[x]$ having these properties:
  \begin{itemize}
    \item Additions of polynomials as defined above
    \item Multiplication of polynomials as defined above
    \item The ring $R$ becomes a subring of $R[x]$ when the elements of $R$ are identifies with the constant polynomials
  \end{itemize}
  \item \textbf{Division with Remainder}: Let $R$ be a ring, $f$ is a monic polynomial, and $g$ is any polynomial, both with coefficients in $R$. There are uniquely determined polynomials $q$ and $r$ in $R[x]$ s.t. $g(x) = f(x)q(x) + r(x)$ where $r$ has degree $\geqslant 0$ and $\leqslant f$
  \begin{itemize}
    \item Division with remainder can be done whenever the leading coefficient of $f$ is a unit
    \item If $g(x)$ is a polynomial in $R[x]$ and $\alpha$ is an element of $R$, the remainder of division of $g(x)$ by $x - \alpha$ is $g(\alpha)$. Thus $x - \alpha$ divides $g$ in $R[x]$ iff $g(\alpha) = 0$
  \end{itemize}
  \item \textbf{Monomial}: a formal product of some variables $x_1, ..., x_n$ of the form
  \begin{center}
    $x_1^{i_1}x_2^{i_2}...x_n^{i_n}$
  \end{center}
  where $i_v$ are non-negative integers.
  \begin{itemize}
    \item \textbf{Degree}: the sum $i_1 + ... + i_n$, sometimes called \textit{total degree}
    \item \textbf{Multi-index}: an $n$-tuple that can be represented with vector notation e.g.  $i = (i_1, ... i_n)$.
    \item A monomial can be written as $x^i$ ($= x_1^{i_1}x_2^{i_2} ... x_n^{i_n}$) using multi-index form
    \item The monomial $x^0$ is denoted by 1
  \end{itemize}
  \item With multi-index notation, a polynomial $f(x) = f(x_1, ..., x_n)$ can be written in exactly one way in the form
  \begin{center}
    $f(x) = \sum\limits_ia_ix^i$
  \end{center}
  where $i$ runs through all multi-indices $(i_1, ..., i_n)$, the coefficients $a_i$ are in $R$ and only finitely many of these coefficients are not 0.
  \item \textbf{Homogeneous Polynomial}: A polynomial in which all monomials with nonzero coefficients have degree $d$
\end{itemize}

\subsection*{1.3: Homomorphisms and Ideals}
\begin{itemize}
  \item \textbf{Ring Homomorphism}: A \textit{ring homomorphism} $\phi: R \rightarrow R'$
\end{itemize}


\iffalse
\subsection*{Problem 1.1}
\textbf{Prove that 7 + $\sqrt[3]{2}$ and $\sqrt{3}$ + $\sqrt{-5}$ are algeabraic numbers.}
\begin{proof}
Since p is not in M, it is not between or equal to a and b, and therefore any interval that contains p is not within M.
\end{proof}
\fi

\end{document}
